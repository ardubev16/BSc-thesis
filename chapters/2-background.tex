\chapter{Background}
\label{cha:background}

In questo capitolo verrà descritta la piattaforma oggetto di studio, saranno inoltre
introdotte delle definizioni e concetti generali riguardanti la Threat Intelligence
e la Open Source Intelligence.

\section[SATAYO]{SATAYO\cite{satayo}}
\label{sec:satayo}

Come accennato precedentemente, SATAYO è un servizio che colleziona, aggrega e
presenta informazioni di natura OSINT in modo semi automatico. È disponibile in modalità
\textit{One Time}, in cui un cliente può richiedere uno scan della propria esposizione,
il quale sarà consultabile per qualche settimana; \textit{SaaS (Software as a Service)},
in cui viene continuamente monitorata l'esposizione online di un cliente con la possibilità
di consultare direttamente la piattaforma; infine in modalità \textit{SaaS \&
Managed} in cui, oltre all'accesso diretto come in modalità SaaS, il cliente ha anche
a disposizione un team di specialisti che analizzano le evidenze collezionate e forniscono
direttamente istruzioni per eventuali mitigazioni necessarie. Le ricerche di
SATAYO sono quasi completamente automatizzate, vengono infatti svolte le azioni necessarie
per il collezionamento delle risorse senza intervento manuale: per esempio
partendo da un dominio, automaticamente vengono trovati i suoi sotto domini, che
a loro volta verranno analizzati per testare eventuali servizi esposti. Contrariamente
azioni come l'inserimento di metadati del cliente per perfezionare le ricerche,
e il controllo qualità a fine ricerca, previa pubblicazione, vengono svolte
manualmente da un analista. SATAYO non è limitato alle risorse presenti nel \textit{Surface
Web}\cite{Kavallieros2021}, esegue infatti ricerche e utilizza informazioni
presenti nel \textit{Deep Web}, ovvero la parte di internet non indicizzata dai
motori di ricerca, e nel \textit{Dark Web} che è accessibile solo tramite
software specifici come \texttt{Tor}\footnote{\url{https://www.torproject.org/}}.
In questa ultima porzione di internet è possibile trovare anche piattaforme in cui
vengono venduti oggetti e software illegali, come può essere l'accesso
privilegiato all'infrastruttura di un cliente.

\section{OSINT}
\label{sec:osint}

L'OSINT\cite{GLASSMAN2012673}, ovvero Open Source INTelligence, è una branca
dell'intelligence che si occupa della ricerca e analisi di dati ottenuti da
fonti accessibili pubblicamente. In generale queste informazioni possono essere ottenute
da risorse di diverso tipo, quali mezzi di comunicazione (giornali, riviste, televisione),
dati governativi pubblici, pubblicazioni accademiche, dati commerciali e,
soprattutto, dall'internet. È proprio quest'ultima la fonte principale
utilizzata da SATAYO, online infatti si trovano molte informazioni che per natura
devono essere pubbliche, ma spesso non ci si presta attenzione. Esempi di queste
informazioni possono essere: record DNS, i quali rivelano molte informazioni riguardanti
il perimetro pubblico di un'organizzazione; account su piattaforme di social
network, quali Linkedin, che riportano in modo piuttosto dettagliato la lista di
dipendenti di un'azienda con i rispettivi ruoli all'interno della stessa; market
nel \textit{Dark Web} che mettono in vendita informazioni che possono essere
ritenute pericolose se di dominio pubblico, come log rubati da macchine infette e
accesso privilegiato alla rete aziendale.

\section{Cyber Threat Intelligence}
\label{sec:cti}

La \textit{Cyber Threat Intelligence}\cite{lee2023cyber} rappresenta l'analisi
di intelligence ottenuta tramite fonti OSINT, file di log, analisi del traffico
di rete di un'organizzazione e analisi forensi. Lo scopo della CTI è di studiare
e individuare delle possibili minacce, sia cyber che fisiche, in modo tale da agire
proattivamente agli attacchi che si potrebbero presentare in futuro. Grazie a
questa tecnica per le aziende è possibile individuare quali minacce presentano
un rischio maggiore e agire di conseguenza. Parte della CTI è anche l'individuazione
di \textit{Threat Actors}, persone o gruppi di persone che pongono una minaccia nei
confronti di aziende e organizzazioni. Solitamente si ottengono evidenze di
questi TA tramite ricerche nel deep o dark web, vengono poi correlate le informazioni
ottenute con altri dati riguardanti i clienti per individuare eventuali minacce.