\chapter{Introduzione}
\label{cha:introduction}

Questo elaborato presenta ed analizza l'esperienza affrontata durante il
tirocinio formativo presso l'azienda Würth Phoenix. Würth Phoenix\footnote{\url{https://www.wuerth-phoenix.com/azienda/}}
è un'azienda di consulenza IT fondata a Bolzano nel 2000 e facente parte del
Gruppo Würth. Ad oggi è formata oltre 200 dipendenti e da tre sedi, di cui la
principale a Bolzano e altre due a Milano e Roma. L'azienda offre servizi e consulenza
IT in ambito di Business Applications, IT Systems Management, Service Management
e Cyber Security. Il tirocinio è stato svolto all'interno del team SEC4U, il quale
si occupa di soluzioni di sicurezza informatica, mettendo a disposizione i
seguenti servizi\footnote{\url{https://www.wuerth-phoenix.com/cyber-security/}}:

\begin{itemize}
  \item \textbf{Offensive Services:} consistono principalmente in Penetration
    Testing e Red Teaming, attività che hanno lo scopo di testare in modo attivo
    l'infrastruttura aziendale per trovare vulnerabilità che potrebbero essere sfruttate
    da individui malevoli. Queste attività vengono svolte simulando gli attacchi
    effetuati da veri Hacker, talvolta con l'ausilio di campagne di phishing, in
    modo tale da trovare tutte le falle all'interno dell'infrastruttura di un
    cliente;

  \item \textbf{Defensive Services:} sono offerti servizi di monitoraggio dell'infrastruttura
    tramite il SOC (Security Operation Center) Attacker Centric attivo 24/7.
    Questo servizio è possibile grazie all'utilizzo di NetEye Cloud, su cui vengono
    collezionati i log e gli eventi di sicurezza prodotti dall'infrastruttura del
    cliente, permettendo agli analisti di analizzare tali eventi ed
    eventualmente, contattare il cliente in caso di traffico o azioni anomale;

  \item \textbf{Cyber Threat Intelligence:} consiste nel monitorare l'esposizione
    online di un'organizzazione e di scoprire anticipatamente la presenza di
    minacce. Questo servizio è offerto grazie a SATAYO, una piattaforma di OSINT
    e Threat Intelligence automatizzata, completamente sviluppata internamente al
    team SEC4U.
\end{itemize}

Quest'ultima, nello specifico, è stato l'argomento principale del lavoro svolto
durante il tirocinio, il quale ha comportato un'analisi e revisione dell'infrastruttura
backend di SATAYO. In seguito ad un'analisi sulla quantità di asset da
monitorare all'interno della piattaforma, è stato infatti ritenuto necessario una
profonda reimplementazione del sistema che si occupa del collezionamento delle informazioni
OSINT, la quale verrà descritta approfonditamente nei prossimi capitoli.